\cvsection{Projects(Github)}
\begin{cventries}
\cventry
    {Technologies: \quad Nodejs, JSON-RPC2.0
    }
    {Oneledger Chain SDK}
    {}
    {Apr. 2019 - May. 2019}
    {
      \begin{cvitems}
        \item {Oneledger Chain SDK is the client side layer which allows clients send RPC requests to make a call or a transaction.}
        \item {This SDK uses JSON-RPC2.0 as request format, and handles response using Promise.}
      \end{cvitems}
    }

\cventry
    {Technologies: \quad Ed25519, RIPEMD160, Nodejs
    }
    {Oneledger Wallet}
    {}
    {May. 2019 - Now}
    {
      \begin{cvitems}
        \item {Oneledger Wallet allows clients to create new keypairs and sign RawTx.}
        \item {This Wallet is using Ed25519 as key algorithm and using RIPEMD160 as hash algorithm.}
        \item {This Wallet allows clients to generate RawTx offline which means without contacting any Oneledger fullnode, and sign txs signature offline.}
      \end{cvitems}
    }
    
\cventry
    {Technologies: \quad Go, Tendermint
    }
    {Oneledger Chain}
    {}
    {May. 2019 - Now}
    {
      \begin{cvitems}
        \item {Oneledger chain will be one of the main Blockchain networks which allows enterprise level interoperability.}
        \item {Using Tendermint as low level consensus algorithm component, Oneledger chain builds transaction layer and distributed database layer on fullnode and explorer nodes.}
        \item {I contributed in ApplyValidator and Fee features.}
      \end{cvitems}
    }

\cventry
    {Technologies: \quad Truffle, Parity, AWS
    }
    {Skyquark Aristotle Chain}
    {}
    {May. 2018 - Apr. 2019}
    {
      \begin{cvitems}
        \item {Aristotle Chain using parity with PoA consensus algorithm generates an EVM based public chain for the foundation of Skyquark Eco-system.}
        \item {Users need to sign up Edusphere system to get a wallet address connecting to Aristotle Chain.}
        \item {ETH Pre-fund accounts are transaction validators who are responsible for mining new blocks, validating all transactions.}
        \item {Users could be part of validators and sync full Aristotle Chain blocks to earn more tokens.}
      \end{cvitems}
    }
    
\cventry
    {Technologies: \quad Nodejs, KoA, MongoDB, LevelDB, web3.js, truffle, AWS, Postman
    }
    {Skyquark Crowdsourcing System}
    {}
    {Sept. 2018 - Apr. 2019}
    {
      \begin{cvitems}
        \item {Crowdsourcing system is the extension of Nodejs back-end web server between Edusphere system and Aristotle Chain.}
        \item {It's also the Aristotle chain user address, chain nodes and smart contracts   management center.}
        \item {It connects with MongoDB as main business logic database and LevelDB as users reputation and other behaviors databases.}
        \item {It communicates with Aristotle Chain through Web3.js API.}
        \item {It provides APIs to the front-end edusphere system.}
        \item {It provides users a revenue auto-distribution system based on the Aristotle Token mechanism without signing any paper or electrical revenue sharing contracts}
        \item {Users behaviors on edusphere.io will be used to calculate the reputation and token rewards and the smart contracts deployed on Aristotle Chain will be triggered to make reward transactions}
      \end{cvitems}
    }

  \cventry
    {Technologies: \quad Solidity, Truffle, Remix IDE, Ethereum Wallet, OpenZeppelin, Quantstamp
    }
    {Skyquark ICO Smart Contract for Aristotle Token}
    {}
    {Aug. 2018 - Oct. 2018}
    {
      \begin{cvitems} 
        \item {This ICO smart contract has four components: Secure math, ownership control, pause, ERC-20 implementation.}
        \item {It has been tested by OpenZeppelin and Quantstamp.}
        \item {It was deployed on Ethereum mainnet but tested on Rinkeby.}
        \item {Aristotle Token is able to be transferred between Ethereum mainnet and Aristotle chain using parity bridge protocol(developing).}
      \end{cvitems}
    }

 
  \cventry
    {Technologies: \quad Truffle, Solidity, Web3, NodeJs, IPFS, Webpack
    }
    {Skyquark Intellectual Property Protection dApp}
    {}
    {June. 2018 - Sept. 2018}
    {
      \begin{cvitems}
        \item {IP Protection dApp developed on Rinkeby testnet first and deployed to Aristotle Chain after.}
        \item {It has four different layers: chain layer, smart contract layer, web3(NodeJs) layer and UI layer}
        \item {For the web part, it needs a web server like AWS to host the all front-end resource files(or IPFS) and web3 will connect to blockchain provider to interact with smart contracts and send transactions to the chain.}
        \item {For all large size documents that users submit, they are all stored on the IPFS, only hash value will be stored into blockchain.}
      \end{cvitems}
    }

  \cventry
    {Technologies: \quad Android, adb shell monkey, Android system permission, Android system sign, Git}
    {Monkey}
    {}
    {June. 2017 - Sept. 2017}
    {
      \begin{cvitems}
        \item {This is an Android application for BlackBerry Spark Day.}
        \item {This app encapsulates adb shell monkey command for BlackBerry internal pressure and performance testing.}
        \item {Granting Android system permission and system sign to run sudo commands.}
        \item {Running monkey command on target apps and generating the testing report for analyzing.}
      \end{cvitems}
    }
  
  \cventry
    {Technologies: \quad Android, Google Map API, Geolocation, AWS, MySQL, Web Server, Git}
    {Waterloo Discovery}
    {}
    {Feb. 2017 - Feb. 2017}
    {
      \begin{cvitems}
        \item {Developed an Android application about exploring the city of Waterloo for new comers such as international students and travelers in 2 people's team.}
        \item {Using Google Map API to locate users, then calculate the distance.}
        \item {Using Google Nearby resources.}
        \item {Awarded Top 5 Hackathon project.}
      \end{cvitems}
    }
    
    % \cventry
    % {Technologies: \quad Android, WebSocket, Multi-threading, Concurrency, MySQL, Git}
    % {Instant Message system}
    % {}
    % {Oct. 2016 - Jan. 2017}
    % {
    %   \begin{cvitems}
    %     \item {Developed an IM Android application for users to join in groups and chat.}
    %     \item {Developed both client and server. The service was built on WebSocket.}
    %   \end{cvitems}
    % }
    
    % \cventry
    % {Technologies: \quad JavaEE, Struts2, Spring3, Hibernate4, Maven, Nexus, Tomcat, MySQL, Design patterns, MVC, Git}
    % {FreeX(Backend)}
    % {}
    % {Oct. 2016 - Dec. 2016}
    % {
    %   \begin{cvitems}
    %     \item {Designed the architecture and developed the web server based on JavaEE for currency exchange Android App.}
    %     \item {Using Factory pattern and Abstract Data Access Object class in Data persistence layer to improve scalability.}
    %     \item {Using transaction management in Spring to ensure atomicity of each transaction.}
    %     \item {Using Session to identify clients and JSON to communicate with clients.}
    %     \item {Designed the database.}
    %   \end{cvitems}
    % }
    
    % \cventry
    % {Technologies: \quad JavaEE, Servlet, DBCP, Java 2D Graphics, Tomcat, MySQL}
    % {Hospital Invoice printing System}
    % {}
    % {Sept. 2016 - Oct. 2016}
    % {
    %   \begin{cvitems}
    %     \item {Designed and developed a web server system based on JavaEE(Servlet, JDBC, Java 2D Graphics) for hospital invoice printing System.}
    %     \item {Using Servlet to distribute requests and responses.}
    %     \item {Using JDBC to persist data into MySQL Database.}
    %     \item {Using Java 2D Graphics API to print out the data in a changeable format}
    %     \item {Using JavaScript and jQuery to develop the frontend for searching, sorting and modifying data}
    %     \item {Designed the database.}
    %   \end{cvitems}
    % }
    
    %  \cventry
    % {Technologies: \quad  Android, SQLite}
    % {Multiple Choice}
    % {}
    % {Aug. 2016 - Oct. 2016}
    % {
    %   \begin{cvitems}
    %     \item {Developed an application to simulate the question asking and answering.}
    %     \item {Two user scenarios for question asking and question answering.}
    %     \item {Timer limited for answering questions and multi-threading controlling.}
    %     \item {Question history system could record each user's behavior and make a ranking}
    %   \end{cvitems}
    % }
    
    % \cventry
    % {Technologies:\quad JavaEE, MVC, Struts2, Hibernate4, Spring-IoC, MySQL, Tomcat, Apache components, AJAX, JavaScript, jQuery, JSP, HTML, CSS, Bootstrap}
    % {Cookie Social Network System }
    % {}
    % {Dec. 2014 - Jun. 2015}
    % {
    %   \begin{cvitems}
    %     \item {Developed a social network system based on JavaEE.}
    %     \item {This project contains several components such as book, music, activity, goods.}
    %     \item {Users are able to follow each other and share different info. They could also make comments just like Facebook}
    %     \item {Designed the database structure and architecture, using SSH frameworks to enhance MVC, Spring was using for Ioc.}
    %     \item {Awarded the Top project scholarship from the Government.}
    %   \end{cvitems}
    % }
    
\end{cventries}
