\cvsection{Projects(Github)}
\begin{cventries}

\cventry
    {Technologies: \quad Go, Tendermint}
    {OneLedger Blockchain Protocol}
    {}
    {Oct. 2019 - Now}
    {
      \begin{cvitems}
        \item {OneLedger blockchain is one of the public Blockchain networks which aims to build enterprise level Interoperability and open source for community to build dApps. (\url{https://github.com/Oneledger/protocol}).}
        \item {Using Tendermint as core consensus engine, OneLedger chain has application(node) layer, RPC layer, transaction verification /distribution layer and chain state distributed database(LevelDB) layer on fullnode and validator node.}
        \item {RPC layer: designed and implemented RPC endpoints for transaction and query.}
        \item {Transaction layer: designed and implemented transaction verification(signature mapping) for all transaction types, transaction fee charge logic, fee collection and reward withdrawal for validators.}
        \item {Distributed Database layer: designed and implemented database schema and store structure for different transaction types and block info.}
        \item {Block scope: designed and implemented block beginner and block end for internal transaction logic.}
        \item {Transactions: OneLedger Domain Service, Governance, Staking\&Delegation, Block Rewards, Multi-currency Transfer, ETH/OLT Interoperability.}
      \end{cvitems}
    }
    
    \cventry
    {Technologies: \quad Go, NodeJS, Redis, PostgreSQL, Bloom Filter, HD wallet, FSM, Ethereum, BitCoin}
    {BitSpawn Deposit Manager}
    {}
    {Nov. 2019 - Nov. 2020}
    {
      \begin{cvitems}
      \item {Designed and implemented Deposit Manager(Golang+NodeJS) server which is Bitspawn Payment Gateway that 1.)accepts deposit by multi-currency(ETH/BTC/USD) and mint SPWN token on internal POA network; 2.) allow withdrawal from SPWN token to multi-currency(ETH/BTC/USD/SPWN)).}
        \item {Cobra CLI: support side features other then deposit/withdrawal: token migration; KMS encryption/decryption; admin mint; smart contract deployment; HD wallet management.}
        \item {Support multiple platforms withdrawal for one currency.}
        \item {Highly configurable: easily switch ON/OFF for deposit/withdrawal currency availability, and support auto or manual process.}
        \item {Finite State Machine design allows future scaliability.}
        \item {Admin customizable Solidity smart contracts ensure the atomicity of both deposit and withdrawal transactions.}  
      \end{cvitems}
    }
    
\cventry
    {Technologies: \quad BitCoin, Ethereum, Typescript, BIP-32, BIP-39, Ed25519, Secp256K1, RIPEMD160, SHA256}
    {OneLedger Hierarchical Deterministic(HD) Wallet}
    {}
    {Jun. 2019 - Apr, 2020}
    {
      \begin{cvitems}
        \item {OneLedger HD Wallet is a JavaScript module that provides ability to generate keypairs and sign Raw Transactions for multiple chains.}
        \item {Support configurable entropy to derive master seed.}
        \item {Support whole wallet backup and recovery from any device by mnemonic.}
        \item {Support BTC, ETH and OLT(OneLedger) keys derivation, address verification and transaction signing.}
        \item {Hide derived master seed and all private key to maximum security.}
        \item {Using hardened extended key path from BIP-39.}
        \item {Easily extendable to support other blockchain key algorithm.}
      \end{cvitems}
    }

\cventry
    {Technologies: \quad Typescript, NodeJS, RPC}
    {OneLedger SDK}
    {}
    {Apr. 2019 - now}
    {
      \begin{cvitems}
        \item {OneLedger SDK is a whole middle layer of JavaScript modules between community developers and OneLedger blockchain protocol which allows developers to query block or transaction info and securely serialise transactions offline and broadcast them.}
        \item {Highly configurable offline transaction serialization and signing workflow which allows community to extend their own transaction types, even using different network.}
        \item {Support different types of transaction broadcasting such as Async, Sync.}
        \item {Support both regular transaction and BTC/ETH Interoperability transaction.}
        \item {Multiple sub-modules design that separates transaction preparation, signing and broadcast into different modules for scaliability.}
        \item {Easily configurable error handler for community to add and handle new errors.}
        \item {Configurable storage usage for different platforms(Web/Chrome Extension/Electron/Native).}
        \item {Support Ledger Device(Cold Wallet) with different connection type such as Bluetooth, USB etc.}
      \end{cvitems}
    }
    
    \cventry
    {Technologies: \quad Go, WebSocket, Tendermint, Tendermint Events, PostgreSQL, SQS, CronJob}
    {OneLedger Blockchain Explorer}
    {}
    {Dec. 2019 - Now}
    {
      \begin{cvitems}
        \item {OneLedger blockchain Explorer is a set of services that watch, synchronize new block from OneLedger blockchain; persist block/transaction data into postgreSQL database; provide restful APIs for querying; send configruable notification for blockchain data alert.(\url{https://oneledger.network})}
        \item {Self-catch-up plus SQS design guarantees the data integrity if Explorer or blockchain node is down or restarted.}
        \item {Micro-services design that data source is writing to SQS, and data handler is reading from SQS.}
        \item {Internal cronjob(goroutines) module maintains the public metrics data.}
        \item {Configurable notification module sends alerts to multiple receivers.}
        \item {Easily extendable transaction type design and DB design for new transaction types.}
        \item {Provide web socket for real-time blockchain info.}
        \item {Provide APIs for community developers to query blocks and transactions info.}
      \end{cvitems}
    }

\cventry
    {Technologies: \quad Truffle, Parity, AWS
    }
    {Skyquark Aristotle Chain}
    {}
    {May. 2018 - Apr. 2019}
    {
      \begin{cvitems}
        \item {Aristotle Chain using parity with PoA consensus algorithm generates an EVM based public chain for the foundation of Skyquark Eco-system.}
        \item {Users need to sign up Edusphere system to get a wallet address connecting to Aristotle Chain.}
        \item {ETH Pre-fund accounts are transaction validators who are responsible for mining new blocks, validating all transactions.}
        \item {Users could be part of validators and sync full Aristotle Chain blocks to earn more tokens.}
      \end{cvitems}
    }
    
\cventry
    {Technologies: \quad Nodejs, KoA, MongoDB, LevelDB, web3.js, truffle, AWS, Postman
    }
    {Skyquark Crowdsourcing System}
    {}
    {Sept. 2018 - Apr. 2019}
    {
      \begin{cvitems}
        \item {Crowdsourcing system is the extension of Nodejs back-end web server between Edusphere system and Aristotle Chain.}
        \item {It's also the Aristotle chain user address, chain nodes and smart contracts   management center.}
        \item {It connects with MongoDB as main business logic database and LevelDB as users reputation and other behaviors databases.}
        \item {It communicates with Aristotle Chain through Web3.js API.}
        \item {It provides APIs to the front-end edusphere system.}
        \item {It provides users a revenue auto-distribution system based on the Aristotle Token mechanism without signing any paper or electrical revenue sharing contracts}
        \item {Users behaviors on edusphere.io will be used to calculate the reputation and token rewards and the smart contracts deployed on Aristotle Chain will be triggered to make reward transactions}
      \end{cvitems}
    }

%   \cventry
%     {Technologies: \quad Solidity, Truffle, Remix IDE, Ethereum Wallet, OpenZeppelin, Quantstamp
%     }
%     {Skyquark ICO Smart Contract for Aristotle Token}
%     {}
%     {Aug. 2018 - Oct. 2018}
%     {
%       \begin{cvitems} 
%         \item {This ICO smart contract has four components: Secure math, ownership control, pause, ERC-20 implementation.}
%         \item {It has been tested by OpenZeppelin and Quantstamp.}
%         \item {It was deployed on Ethereum mainnet but tested on Rinkeby.}
%         \item {Aristotle Token is able to be transferred between Ethereum mainnet and Aristotle chain using parity bridge protocol(developing).}
%       \end{cvitems}
%     }

 
  \cventry
    {Technologies: \quad Truffle, Solidity, Web3, NodeJs, IPFS, Webpack
    }
    {Skyquark Intellectual Property Protection dApp}
    {}
    {Jun. 2018 - Sept. 2018}
    {
      \begin{cvitems}
        \item {IP Protection dApp developed on Rinkeby testnet first and deployed to Aristotle Chain after.}
        \item {It has four different layers: chain layer, smart contract layer, web3(NodeJs) layer and UI layer}
        \item {For the web part, it needs a web server like AWS to host the all front-end resource files(or IPFS) and web3 will connect to blockchain provider to interact with smart contracts and send transactions to the chain.}
        \item {For all large size documents that users submit, they are all stored on the IPFS, only hash value will be stored into blockchain.}
      \end{cvitems}
    }

  \cventry
    {Technologies: \quad Android, adb shell monkey, Android system permission, Android system sign, Git}
    {Monkey}
    {}
    {Jun. 2017 - Sept. 2017}
    {
      \begin{cvitems}
        \item {This is an Android application for BlackBerry Spark Day.}
        \item {This app encapsulates adb shell monkey command for BlackBerry internal pressure and performance testing.}
        \item {Granting Android system permission and system sign to run sudo commands.}
        \item {Running monkey command on target apps and generating the testing report for analyzing.}
      \end{cvitems}
    }
  
  \cventry
    {Technologies: \quad Android, Google Map API, Geolocation, AWS, MySQL, Web Server, Git}
    {Waterloo Discovery}
    {}
    {Feb. 2017 - Feb. 2017}
    {
      \begin{cvitems}
        \item {Developed an Android application about exploring the city of Waterloo for new comers such as international students and travelers in 2 people's team.}
        \item {Using Google Map API to locate users, then calculate the distance.}
        \item {Using Google Nearby resources.}
        \item {Awarded Top 5 Hackathon project.}
      \end{cvitems}
    }
    
    % \cventry
    % {Technologies: \quad Android, WebSocket, Multi-threading, Concurrency, MySQL, Git}
    % {Instant Message system}
    % {}
    % {Oct. 2016 - Jan. 2017}
    % {
    %   \begin{cvitems}
    %     \item {Developed an IM Android application for users to join in groups and chat.}
    %     \item {Developed both client and server. The service was built on WebSocket.}
    %   \end{cvitems}
    % }
    
    % \cventry
    % {Technologies: \quad JavaEE, Struts2, Spring3, Hibernate4, Maven, Nexus, Tomcat, MySQL, Design patterns, MVC, Git}
    % {FreeX(Backend)}
    % {}
    % {Oct. 2016 - Dec. 2016}
    % {
    %   \begin{cvitems}
    %     \item {Designed the architecture and developed the web server based on JavaEE for currency exchange Android App.}
    %     \item {Using Factory pattern and Abstract Data Access Object class in Data persistence layer to improve scalability.}
    %     \item {Using transaction management in Spring to ensure atomicity of each transaction.}
    %     \item {Using Session to identify clients and JSON to communicate with clients.}
    %     \item {Designed the database.}
    %   \end{cvitems}
    % }
    
    % \cventry
    % {Technologies: \quad JavaEE, Servlet, DBCP, Java 2D Graphics, Tomcat, MySQL}
    % {Hospital Invoice printing System}
    % {}
    % {Sept. 2016 - Oct. 2016}
    % {
    %   \begin{cvitems}
    %     \item {Designed and developed a web server system based on JavaEE(Servlet, JDBC, Java 2D Graphics) for hospital invoice printing System.}
    %     \item {Using Servlet to distribute requests and responses.}
    %     \item {Using JDBC to persist data into MySQL Database.}
    %     \item {Using Java 2D Graphics API to print out the data in a changeable format}
    %     \item {Using JavaScript and jQuery to develop the frontend for searching, sorting and modifying data}
    %     \item {Designed the database.}
    %   \end{cvitems}
    % }
    
    %  \cventry
    % {Technologies: \quad  Android, SQLite}
    % {Multiple Choice}
    % {}
    % {Aug. 2016 - Oct. 2016}
    % {
    %   \begin{cvitems}
    %     \item {Developed an application to simulate the question asking and answering.}
    %     \item {Two user scenarios for question asking and question answering.}
    %     \item {Timer limited for answering questions and multi-threading controlling.}
    %     \item {Question history system could record each user's behavior and make a ranking}
    %   \end{cvitems}
    % }
    
    % \cventry
    % {Technologies:\quad JavaEE, MVC, Struts2, Hibernate4, Spring-IoC, MySQL, Tomcat, Apache components, AJAX, JavaScript, jQuery, JSP, HTML, CSS, Bootstrap}
    % {Cookie Social Network System }
    % {}
    % {Dec. 2014 - Jun. 2015}
    % {
    %   \begin{cvitems}
    %     \item {Developed a social network system based on JavaEE.}
    %     \item {This project contains several components such as book, music, activity, goods.}
    %     \item {Users are able to follow each other and share different info. They could also make comments just like Facebook}
    %     \item {Designed the database structure and architecture, using SSH frameworks to enhance MVC, Spring was using for Ioc.}
    %     \item {Awarded the Top project scholarship from the Government.}
    %   \end{cvitems}
    % }
    
\end{cventries}
